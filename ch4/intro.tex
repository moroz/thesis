\section{Analysis of the introduction}

The novel \textit{Marta} begins with an introductory monologue, describing the importance of love in the life of a woman.
% The introduction precedes the life story of the protagonist, Marta Świcka.
The introduction is written from the point of view of a third-person omniscient narrator.
It is not clear whether the narrator can be identified with the author of the novel, however, the opinions presented in the introduction can be interpreted as belonging to the author.
In this section, the narrator takes issue with the novel \textit{Albina} by Mr.\ Jan Zachariasiewicz (1825--1906).
Zachariasiewicz is attributed with stating that the reason women suffer, either on the moral or the physical level, is because they lack true love for a man, and that for a woman, the act of marriage is always motivated by pure calculation.
The narrator considers this statement to be unjust and argues that the whole existence of a woman is built around the concept of love:
Starting from early childhood, young girls yearn to grow up and have the honor to meet their destined life partner.
In some cases, their wishes are granted, and they end up marrying a man they love at the church and living happily ever after.
In many other cases, something along the way goes wrong and the woman has to live a sinful life of suffering and hunger.
This introduction makes ground for the story of Marta Świcka, whose life goes astray in a lot of ways.

In the source text, the introduction is written in a lofty, poetic style with archaic wording.
Zamenhof managed to translate the introduction with a high degree of equivalence.
In this part of the translation, the Author did not manage to find any deviations in meaning from the source text.
For a modern reader versed in both Polish and Esperanto, the translation may be even easier to understand than the source text.
This is due to the fact that the Polish language has evolved significantly over the past hundred of years, while the fundamental rules of Esperanto remained intact.

% Syntactic parallelism

% \begin{displayquote}
%   Życie kobiety to wiecznie gorejący płomień miłości --- powiadają \textbf{jedni}.
%   Życie kobiety to zaparcie się --- twierdzą \textbf{inni}.
%   Życie kobiety to macierzyństwo --- wołają \textbf{tamci}.
%   Życie kobiety to igraszka --- żartują \textbf{inni jeszcze}.
% \end{displayquote}

% \begin{displayquote}
%   La vivo de virino estas eterne brulanta flamo de amo, diras \textbf{unuj}. La vivo de virino estas sinoferado, certigas \textbf{aliaj}. La vivo de virino estas patrineco, krias \textbf{parto da homoj}. La vivo de virino estas amuziĝado, ŝercas \textbf{aliaj}.
% \end{displayquote}

