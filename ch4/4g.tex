%%% TeX-master: "../main"

\section{Analysis of chapter seven}

\subsection{Summary of the plot of chapter seven}

Mr.\ Aleksander Łącki, the cousin of Mrs.\ Rudzińska, goes to the sewing parlor of Mrs.\ Szwejcowa to find Marta.
The visit does not go unnoticed for her coworkers and for Mrs.\ Szwejcowa, and prompts the latter to make further inquiries about their relationship.
She mentions that she had seen Marta the previous day in front of the Church of the Holy Cross, talking to Aleksander and Karolina, and revealing that she knew them both.
Aleksander was notorious for his machismo and had broken the heart of one of Mrs.\ Szwejcowa's daughters, whereas Karolina had previously worked at the same sewing parlor and had been fired for her unseemly lifestyle.
To make matters even worse, the daughter of Mrs.\ Szwejcowa, an ugly girl who in spite of her relative unattractivity had been approached by Mr.\ Łącki before, insinuates that Marta must be in good terms with him, and that they probably take walks together every day.
Mrs.\ Szwejcowa states that as the owner of the business she cannot tolerate that her workers be seen in public with a widely known playboy and pressures Marta to leave.
Marta reacts with pride and resigns from the job of her own volition, throwing away her last means of livelihood.

After leaving the sewing parlor, in the gate of the edifice where the parlor was located, she encounters Mr.\ Łącki, talking to another young man.
She tries to avoid his gaze, but fails to go unnoticed.
Aleksander greets Marta, who reproaches him for playing with her feelings and depriving her of her livelihood with his folly.
She criticizes his acts by quoting a fable by a Polish poet from the Enlightenment period, Ignacy Krasicki, titled \textit{Children and Frog}:

\begin{displayquote}
    Chłopcy, przestańcie, bo się źle bawicie!\\
    Dla was to jest igraszką, nam idzie o życie\footnote{%
      From Krasicki. In a loose translation by the author: ,,Boys, stop, because you are playing wrong! For you, it is a game, for us it is about life.''}
\end{displayquote}

Having said these words, she leaves Aleksander dumbfounded, standing in the gate.
For a brief moment, Aleksander seems to ponder the consequences of his actions towards the poor widow, but soon forgets about Marta when he sees a young, fashionable girl walking in through the gate.
It is Miss Eleonora Szwejcówna, one of the daughters of Mrs.\ Szwejcowa.
She encourages Aleksander to come visit their home more often.
Miss Eleonora enters the courtyard of the edifice, but Mr.\ Łącki does not follow her there, thinking it would be improper for the seamstresses to see him walking around with the heiress.

The chapter ends with an exclamation paraphrasing a monologue from \textit{Forefathers' Eve}, a poetic drama by Adam Mickiewicz. The original phrase read: ,,Kobieto! puchu marny! ty wietrzna istoto!'' (`Woman! You wretched fluff! You fickle creature!'), and was adjusted by Orzeszkowa to refer to men.
