\section{Analysis of chapter I}

This section describes the comparative analysis of chapter I, as delimited in the source text, and the corresponding part of the target text.

\subsection{Summary of the plot of chapter one}
The first chapter of the novel describes the beginning of a new phase in the life of the protagonist, Marta Świcka and her daughter, Janina, following the death of Marta's husband.
Mrs.\ Świcka, having lost her only supporter, is compelled to move out of a big and comfortable apartment in Graniczna street, into a small, single room in the attic of a much humbler edifice in Piwna street, in the Old Town.
The woman, who has never needed to work for money in her life, needs to find a job to support herself and her four-year-old daughter.

The chapter starts by presenting the circumstances in which the protagonist leaves her old apartment, surrounded by porters moving expensive furniture out of the building, taking away the last remnants of a life that the impoverished and incomplete family can no longer afford.
After all furniture and belongings have been taken away, Marta goes to her new rented room.
On the way, she is accompanied by her daughter and her former servant girl, Zofia, which subsequently leaves for her new masters' place.

The chapter continues with a description of the protagonist's nightly errand to buy food for her child, in an atmosphere of fear and uncertainty.
It is the first time she needs to go outside on her own at night.
Afraid to leave, she encounters two people, the caretaker of her edifice, and a woman whom she guessed to be the caretaker's wife.
She pleads both of them to run the errand for her, but both refuse, seeing the widow's poor financial condition.
Finally, the protagonist goes to a shop to buy food, overcoming her fears.
On the way home, she is approached by a stranger humming a frivolous song, who shouts at her to slow down and ``go for a walk'' with him.
In spite of all the perils, she manages to return home to her daughter, make tea, and put her daughter to sleep.
The chapter ends with the protagonist sitting at the fireplace and thinking about her new way of life.

\subsection{General appraisal of the translation of chapter I}
In this part of the translation, Zamenhof managed to preserve both the precise meaning of most of the words and their literary value. The text is easy to read and understand for a reader versed in Esperanto.

There are, however, slight omissions and changes which do modify the perception of the translation.
An important detail of the translation apparent from this section of the translation is that the names of streets in Warsaw: Graniczna and Piwna, are left in their original Polish spelling, meaningless to a reader not versed in any Slavic language. % TODO: Are all of them left in Polish?
As a result, the name ``ulica Graniczna'' (lit. `[the street] at the boundary'), has been rendered phonetically in the Chinese translation as Gelanazhina Street (格拉納支納街 \pinyin{Gélānàzhīnà jiē}) which not only does not convey any meaning, but is also a mispronunciation of the Polish name%
\footnote{This word is pronounced \ipa{[ɡrãˈɲiʧ̑na]}, so a closer rendition into Chinese would be Gelanizhina Street (格拉尼支納街 \pinyin{Gēlānízhīnà jiē}).}.
while the name ``Piwna'' means ``the beer street.''
One could argue that the meanings of those names do not really add any value to the story, but at least in the case of street names, a phonetic rendition does result in a loss of meaning.

\subsection{Specific differences between texts}
In the whole section, the Polish verb \textit{szeptać} (`to whisper') is always translated into Esperanto as \textit{mallaŭte paroli} (`to speak quietly').

\begin{displayquote}
--- Mamo! --- szepnęła dziewczynka --- patrz! biurko ojca!
\end{displayquote}

\begin{displayquote}
--- Panjo! --- mallaŭte diris la knabineto, --- la skribtablo de la patro!
\end{displayquote}

Both expressions convey similar meaning and though not fully equivalent, this change is unlikely to result in mistranslations in the Chinese text.

In the following passage, a phrase using an active verb is restructured into passive voice:

\begin{displayquote}
Znać było, że przedmioty te, \textbf{z którymi rozstawała się widocznie}, posiadały dla niej nie tylko materialną cenę;
\end{displayquote}

It was apparent that these objects, \textbf{with which she was visibly parting}, possesed for her not only material price;

\begin{displayquote}
Oni povis rimarki, ke tiuj objektoj, \textbf{kiuj videble estis forprenataj de ŝi}, havis por ŝi valoron ne sole materialan;
\end{displayquote}

One could notice that these objects, \textbf{which visibly were being taken away from her}, possesed for her not only material price;

\subsection{Omissions}

Certain sentences or words in the source text were omitted in the translation.
This sentence describes the grand piano that Marta had to sell:

\begin{displayquote}
Po biurku ukazał się na dziedzińcu \textbf{ładny kralowski fortepian}, ale kobieta w żałobie obojętniej już za nim wzrokiem powiodła.
\end{displayquote}

After the desk, in the courtyard appeared \textbf{a good-looking Krall \& Seider grand piano}, but the woman in a mourning dress followed it with her eyes more indifferently.

In the translation, the expression \textit{ładny kralowski fortepian} (`a good-looking grand piano made by the company Krall \& Seider') has been translated as \textit{bela fortepiano} (`a beautiful grand piano'), omitting the indication of origin altogether:

\begin{displayquote}
Post la skribtablo aperis sur la korto \textbf{bela fortepiano}, sed la virino en funebra vesto akompanis ĝin jam per rigardo pli indiferenta.
\end{displayquote}

After the desk appeared in the courtyard \textbf{a good-looking grand piano}, but the woman in a mourning dress followed it with a more indifferent glance.

Krall i Seider was a maker of pianofortes in Poland during the life and times of Orzeszkowa, but the company name was unlikely to be recognized abroad.
Therefore, the omission seems reasonable and was not likely to result in mistranslations (Museum of Art in Łódź).

When Marta's daughter notices her bed being taken away, she shouts out in dismay:

\begin{displayquote}
--- Łóżeczko moje, mamo! --- zawołała dziewczynka --- ludzie ci i łóżeczko moje zabierają, i tę kołderkę, którąś mi sama zrobiła! \textbf{Ja nie chcę, aby oni to zabierali!} Odbierz, mamo, od nich łóżeczko moje i kołderkę.
\end{displayquote}

``My little bed, mom!'' shouted the little girl. ``Those people are even taking my little bed, and the little quilt that you made for me yourself! I don't want them to take it away! Mom, take back my little bed and my little quilt from them.''

In the translation, the sentence \textit{Ja nie chcę, aby oni to zabierali!} (I do not want them to take it away) is omitted altogether:

\begin{displayquote}
--- Mia liteto, panjo! --- ekkriis la knabineto --- tiuj homoj forprenas ankaŭ mian liteton, kaj ankaŭ tiun kovrileton, kiun vi mem al mi faris! Reprenu, panjo, de ili mian liteton kaj la kovrileton.
\end{displayquote}

``My little bed, mom!'' shouted the little girl. ``Those people are even taking away my little bed, and also that little quilt which you made for me yourself. Mom, take back from them my little bed and my little quilt.''

\subsection{Nativization of given names, archaisms, and bad grammar}

Let us take a look at the parting words that the protagonist, Marta, exchanged with her servant girl, Zosia:

\begin{displayquote}
--- Dziękuję ci, \textbf{Zosiu} --- rzekła cicho --- byłaś dla mnie bardzo dobrą.

--- \textbf{Pani} to \textbf{byłaś} zawsze dobrą dla mnie --- zawołała dziewczyna --- służyłam u pani cztery lata i nigdzie nie było mi i nie będzie już \textbf{lepiej jak u pani}.
\end{displayquote}

These words were thus rendered in the target text:

\begin{displayquote}
--- Mi dankas vin, \textbf{Sonjo}, ŝi diris mallaŭte, --- vi estis por mi tre bona.

--- Sinjorino, \textbf{vi} ĉiam \textbf{estis} bona por mi, --- ekkriis la knabino, --- mi servis ĉe vi kvar jarojn, kaj nenie estis al mi nek iam estos \textbf{pli bone, ol ĉe vi}.
\end{displayquote}

In the source text, the protagonist calls the maiden ``Zosia,'' which is a diminutive form of the name Zofia.
Although there is no clear rule regarding the forming of diminutive forms of names in Esperanto, Zamenhof nativized the name as ``Sonjo,'' a cognate of Russian \textit{Sonya} (Russian: Соня), which in turn is the diminutive form of the name \textit{Sofiya} (Russian: София).
This is in line with other common diminutives, such as Panjo (`Mommy', diminutive form of \textit{patrino} `mother') and Pa\^cjo (`Daddy', diminutive form of \textit{patro} `father').
Such a rendition of the name is as close as one can get when nativising this particular name.

% ``Thank you, Sophie,'' she said quietly, ``you have been very good to me.''

% ``Milady, you have always been good to me,'' cried the maiden, ``I have served you for four years, and nowhere has ever been, nowhere will ever be better than with you.''

Another thing worth noticing in this fragment, as well as many other places throughout the source text, is that the source text contains archaic and regional grammar and vocabulary that would be considered incorrect in modern Polish. % Znaleźć źródło!
For instance, according to the rules of modern Polish grammar, when using the honorific form \textit{Pani} (`Milady; Mistress; Madam') or \textit{Pan} (`Sir; Mister'), the verb should be conjugated in the third person singular form (as in: \textit{Pani była}).
In the source text, however, the second person singular is used (\textit{Pani byłaś}).
Zosia, the maiden, uses the conjunction % sprawdzić, co to za część mowy
\textit{jak} (`as; how') in comparisons (\textit{lepiej jak u pani}), rather than \textit{niż} (`than', as in: \textit{lepiej, niż u pani}).
Nowadays, this is considered a colloquialism and is incorrect according to formal grammar. % źródło
However, the grammatical rules of Esperanto are few and simple; there are no honorific forms, and all conjugations take exactly the same form, regardless of the grammatical person.
In Esperanto, it is simply not possible to translate anything with bad grammar without altering the meaning of a sentence.
Therefore the translation fails to convey the grammatical mistakes occuring in the speech of the servant girl.
Bad grammar and archaisms, however, do not convey meaning on their own, therefore the Author thinks it safe to assume that the translation of this fragments still achieves a high degree of equivalence.

\section{Use of Esperanto terms specific to Slavic culture}

It is apparent that Zamenhof's translation work has been facilitated by the existence of specific Esperanto words originating from Slavic culture.
For instance, at the end of the chapter, Marta tells her daughter that she had brought her some bread rolls and that she would prepare a samovar and make tea:

\begin{displayquote}
--- Mi alportis al vi bulkojn, Janjo, kaj nun mi preparos la samovaron kaj pretigos teon.
\end{displayquote}

The word \textit{bulko} (Esperanto: `bread roll') used in this sentence is a cognate of Polish \textit{bułka} and Russian \textit{bulka} ({\cyrfont булка}).
While the word for `bread', Latin-derived \textit{pano}, appeared in Zamenhof (1887), as part of the translation of \textit{Father's Prayer}, the word for `bread roll' only appeared in the multilingual dictionary in Zamenhof (1905) as:

\begin{displayquote}
bulk' pain blanc | manchet loaf | Semmel | булка | bułka.
\end{displayquote}

Bread and bread rolls are a staple food throughout Europe, an everyday object for most early adopters of Esperanto.
Therefore an Esperanto
