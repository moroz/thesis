\chapter{Comparison of \textit{Marta} in Polish and Esperanto}

This chapter makes an attempt to describe the contents of the Polish original of the novel \textit{Marta} and compare it with Ludwik Zamenhof's Esperanto translation.
In the process, it should be possible to identify parts of the translation where equivalence could not be achieved and try to describe the way Zamenhof dealt with these issues in the process of his translation work.

\section{Analysis of the introduction}

The novel \textit{Marta} begins with an introductory monologue, describing the importance of love in the life of a woman.
% The introduction precedes the life story of the protagonist, Marta Świcka.
The introduction is written from the point of view of a third-person omniscient narrator.
It is not clear whether the narrator can be identified with the author of the novel, however, the opinions presented in the introduction can be interpreted as belonging to the author.
In this section, the narrator takes issue with the novel \textit{Albina} by Mr. Jan Zachariasiewicz (1825--1906).
Zachariasiewicz is attributed with stating that the reason women suffer, either on the moral or the physical level, is because they lack true love for a man, and that for a woman, the act of marriage is always motivated by pure calculation.
The narrator considers this statement to be unjust and argues that the whole existence of a woman is built around the concept of love:
Starting from early childhood, young girls yearn to grow up and have the honor to meet their destined life partner.
In some cases, their wishes are granted, and they end up marrying a man they love at the church and living happily ever after.
In many other cases, something along the way goes wrong and the woman has to live a sinful life of suffering and hunger.
This introduction makes ground for the story of Marta Świcka, whose life goes astray in a lot of ways.

In the source text, the introduction is written in a lofty, poetic style with archaic wording.
Zamenhof managed to translate the introduction with a high degree of equivalence.
In this part of the translation, the Author did not manage to find any deviations in meaning from the source text.
For a modern reader versed in both Polish and Esperanto, the translation may be even easier to understand than the source text.
This is due to the fact that the Polish language has evolved significantly over the past hundred of years, while the fundamental rules of Esperanto remained intact.

% Syntactic parallelism

% \begin{displayquote}
%   Życie kobiety to wiecznie gorejący płomień miłości --- powiadają \textbf{jedni}.
%   Życie kobiety to zaparcie się --- twierdzą \textbf{inni}.
%   Życie kobiety to macierzyństwo --- wołają \textbf{tamci}.
%   Życie kobiety to igraszka --- żartują \textbf{inni jeszcze}.
% \end{displayquote}

% \begin{displayquote}
%   La vivo de virino estas eterne brulanta flamo de amo, diras \textbf{unuj}. La vivo de virino estas sinoferado, certigas \textbf{aliaj}. La vivo de virino estas patrineco, krias \textbf{parto da homoj}. La vivo de virino estas amuziĝado, ŝercas \textbf{aliaj}.
% \end{displayquote}

\section{Analysis of chapter one}

The first chapter of the novel describes the beginning of a new phase in the life of the protagonist, Marta Świcka and her daughter, Janina, following the death of Marta's husband.

\subsection{Summary of the plot of chapter one}
Mrs. Świcka, having lost her only supporter, is compelled to move out of a comfortable apartment in Graniczna street---right next to a Baroque public park called Saxon Garden---into a single small room in the attic of a much humbler edifice in Piwna street, in the Old Town.
The woman, who has never worked for money in her life, needs to find a job to support herself and her four-year-old daughter.

The chapter starts by presenting the circumstances in which the protagonist leaves her old apartment, surrounded by porters moving expensive furniture out of the building, taking away the last remnants of a life that the impoverished and incomplete family can no longer afford.
After all furniture and belongings have been taken away, Marta goes to her new rented room.
On the way, she is accompanied by her daughter and her former servant girl, Zofia, which subsequently leaves for her new masters' place.

The chapter continues with a description of the protagonist's nightly errand to buy food for her child, in an atmosphere of fear and uncertainty.
It is the first time when she needs to go outside on her own at night.
Afraid to leave, she encounters two people, the caretaker of her edifice, and a woman whom she guessed to be the caretaker's wife.
She pleads both of them to run the errand for her, but both refuse, seeing the widow's poor financial condition.
Finally, the protagonist goes to a shop to buy food, overcoming her fears.
On the way home, she is approached by a stranger humming a frivolous song, who shouts at her to slow down and ``go for a walk'' with him.
In spite of all the perils, she manages to return home to her daughter, make tea, and put her daughter to sleep.
The chapter ends with the protagonist sitting at the fireplace and thinking about her new way of life.

\subsection{Analysis of the language of the Esperanto translation}

In this part of the translation, Zamenhof managed to preserve both the precise meaning of most of the words and their literary value. The text is easy to read and understand for a reader versed in Esperanto.

There are, however, slight omissions and changes which do modify the perception of the translation.
An important detail of the translation apparent from this section of the translation is that the names of streets in Warsaw: Graniczna and Piwna, are left in their original Polish spelling, meaningless to a reader not versed in any Slavic language. % TODO: Are all of them left in Polish?
The name ``Graniczna'' means ``[the street] at the boundary,'' and has been rendered phonetically in the Chinese translation as Gelanazhina Street (格拉納支納街 \pinyin{Gélānàzhīnà jiē}) which not only does not convey any meaning, but is also a mispronunciation of the Polish name, which is pronounced \ipa{[ɡrãˈɲiʧ̑na]}.
while the name ``Piwna'' means ``the beer street.''
Although the meanings of those names do not really bring any added value to the story, one could argue that a phonetic rendition does result in a loss of meaning.

--- Mamo! --- szepnęła dziewczynka --- patrz! biurko ojca!

--- Panjo! --- mallaŭte diris la knabineto, --- la skribtablo de la patro!

\begin{displayquote}
Znać było, że przedmioty te, z którymi \textbf{rozstawała się widocznie}, posiadały dla niej nie tylko materialną cenę;
\end{displayquote}

\begin{displayquote}
Oni povis rimarki, ke tiuj objektoj, kiuj \textbf{videble estis forprenataj de ŝi}, havis por ŝi valoron ne sole materialan;
\end{displayquote}

In the whole section, the Polish verb \textit{szeptać} (`to whisper') is always translated into Esperanto as \textit{mallaŭte paroli} (`to speak quietly').
Both expressions convey similar meaning and though not fully equivalent, this change is unlikely to result in mistranslations in the Chinese text.

\subsection{Omissions}

Certain sentences or words in the source text were omitted in the translation.
This sentence describes the grand piano that Marta had to sell:

\begin{displayquote}
Po biurku ukazał się na dziedzińcu \textbf{ładny kralowski fortepian}, ale kobieta w żałobie obojętniej już za nim wzrokiem powiodła.
\end{displayquote}

In the translation, the expression \textit{ładny kralowski fortepian} (`a good-looking grand piano made by the company Krall \& Seider') has been translated as \textit{bela fortepiano} (`a beautiful grand piano'), omitting the indication of origin altogether:

\begin{displayquote}
Post la skribtablo aperis sur la korto \textbf{bela fortepiano}, sed la virino en funebra vesto akompanis ĝin jam per rigardo pli indiferenta.
\end{displayquote}

The said company was a popular maker of pianofortes in Poland during the life and times of Orzeszkowa, but the company name was unlikely to be recognized abroad.
Therefore, the omission seems reasonable and was not likely to result in mistranslations.

When Marta's daughter notices her bed being taken away, she shouts out in dismay:

\begin{displayquote}
--- Łóżeczko moje, mamo! --- zawołała dziewczynka --- ludzie ci i łóżeczko moje zabierają, i tę kołderkę, którąś mi sama zrobiła! \textbf{Ja nie chcę, aby oni to zabierali!} Odbierz, mamo, od nich łóżeczko moje i kołderkę.
\end{displayquote}

In the translation, the sentence \textit{Ja nie chcę, aby oni to zabierali!} (I do not want them to take it away) is omitted altogether:

\begin{displayquote}
--- Mia liteto, panjo! --- ekkriis la knabineto --- tiuj homoj forprenas ankaŭ mian liteton, kaj ankaŭ tiun kovrileton, kiun vi mem al mi faris! Reprenu, panjo, de ili mian liteton kaj la kovrileton.
\end{displayquote}

\subsection{Nativization of given names, archaisms, and bad grammar}

Let us take a look at the parting words that the protagonist, Marta, exchanged with her servant girl, Zosia:

\begin{displayquote}
--- Dziękuję ci, \textbf{Zosiu} --- rzekła cicho --- byłaś dla mnie bardzo dobrą.

--- \textbf{Pani} to \textbf{byłaś} zawsze dobrą dla mnie --- zawołała dziewczyna --- służyłam u pani cztery lata i nigdzie nie było mi i nie będzie już \textbf{lepiej jak u pani}.
\end{displayquote}

These words were thus rendered in the Esperanto translation:

\begin{displayquote}
--- Mi dankas vin, \textbf{Sonjo}, ŝi diris mallaŭte, --- vi estis por mi tre bona.

--- Sinjorino, \textbf{vi} ĉiam \textbf{estis} bona por mi, --- ekkriis la knabino, --- mi servis ĉe vi kvar jarojn, kaj nenie estis al mi nek iam estos \textbf{pli bone, ol ĉe vi}.
\end{displayquote}

In the source text, the protagonist calls the maiden ``Zosia,'' which is a diminutive form of the name Zofia.
Although there is no clear rule regarding diminutive names in Esperanto, Zamenhof nativized the name as ``Sonjo,'' which is the Esperanto spelling of ``Sonja,'' the diminutive form of the Russian name ``Sofia.''

% ``Thank you, Sophie,'' she said quietly, ``you have been very good to me.''

% ``Milady, you have always been good to me,'' cried the maiden, ``I have served you for four years, and nowhere has ever been, nowhere will ever be better than with you.''

Another thing worth noticing in this fragment, as well as many other places throughout the source text, is that the dialogues contain archaic forms and grammar that would be considered incorrect in modern Polish. % Znaleźć źródło!
For instance, according to the rules of modern Polish grammar, when using the honorific form \textit{Pani} (`Milady; Mistress; Madam') or \textit{Pan} (`Sir; Mister'), the verb should be conjugated in the third person singular form (as in: \textit{Pani była}).
In the source text, however, the second person singular is used (\textit{Pani byłaś}).
Zosia, the maiden, uses the conjunction % sprawdzić, co to za część mowy
\textit{jak} (`as; how') in comparisons (\textit{lepiej jak u pani}), rather than \textit{niż} (`than', as in: \textit{lepiej, niż u pani}).
This is a colloquialism considered incorrect by formal grammar and very common to this day in the spoken language. % źródło
However, the grammatical rules of Esperanto are few and simple; there is no honorific verb form, and all conjugations take exactly the same form, regardless of the grammatical person.
In Esperanto, it is simply not possible to translate anything with bad grammar without altering the meaning of a sentence.
Therefore the translation fails to convey the grammatical mistakes occuring in the speech of the servant girl.
Bad grammar and archaisms, however, do not convey meaning on their own, therefore the Author thinks it safe to assume that the translation of this fragments still achieves a high degree of equivalence.

\section{Summary of the plot of chapter two}

Then follows a brief biography of the protagonist preceding the beginning of the first chapter.
Born to a family of gentry and married to a low-ranking government official by the name Jan Świcki, she lost first her mother, then her father, and finally her husband.
During her husband's disease, Marta had to spend most of their money and sell off her jewellery to pay for medical costs, which turned out to be useless.

On the next day of the removal, Marta goes to the office of a tutoring broker in an attempt to apply for a job.
At the office, she encounters an English woman and a French woman.
Both are offered very well-paid jobs as governesses staying at the residences of wealthy counts, with very good working conditions.
The French woman is even allowed to live with her little niece.
The broker admits that the French woman does not possess any particular teaching qualifications, but has the benefit of being a foreigner.
After a brief interview with the broker, a middle-aged lady, she finds out that she has little hope of finding a teaching job and staying together with her daughter; she could, however, get a decently paid job if she agreed to leave the child to stay with someone else.
Not willing to part with her daughter, Marta offers to provide entry-level classes in such subjects as geography, history, Polish literature, and drawing.
She finds out that she cannot teach these subjects, as they are taught almost exclusively by men.
The broker admonishes Marta that in their society, the only way a woman could earn a decent living with her work is by attaining mastery in a highly demanded skill, such as a foreign language or music; any elementary skills are not enough to secure a carefree existence.

\section{Analysis of chapter three}
\section{Summary of the plot of chapter three}

The third chapter begins with the description of Marta Świcka, the protagonist, coming home after receiving the good news that she had been offered the position of a French language teacher.
She reacts to the news with a lot of enthusiasm, relieved in the knowledge that she will finally be able to support herself and her daughter through her work.
The news come at just the right time: the protagonist has already spent nearly all of her money, and would soon have no money to buy food and firewood.
The tutoring broker that Marta approached back in chapter two told Marta that she may be able to get even more students and be offered better pay if she proves to be a conscientious and skillful teacher.

The story follows with the description of the first meeting with Jadwiga, the 12-year-old daughter of Mr.\ and Mrs.\ Rudziński.
The mother, Mrs.\ Maria Rudzińska, explains that her daughter has already taken classes with an experienced teacher, a French woman by the name of Mrs.\ Dupont.
The reason they chose to offer the position of French tutor to a Polish woman rather than to another foreigner was to support the locals, who found it much more difficult to find jobs.

During their first class, a male cousin of Maria Rudzińska by the name Aleksander Łącki, a handsome bachelor, notices the new teacher and secretly tells Maria about his affection towards the beautiful widow and asks to be introduced to her.
However, Maria tells her cousin off, asking him not to bother the poor widow with his fickle affection, and telling him to seriously reconsider his attitude towards his personal life and career.

Starting from the first meeting with her new student, Marta instantly realizes that little Jadwiga is in fact much more proficient in the French language than herself.
She has a hard time answering her student's questions.
She decides to study on her own to make up for her lacking knowledge, but soon comes to the bitter realization that the lack in her knowledge is too big to be helped with such last-minute efforts.

After a month of classes, Marta decides to put an end to this relationship and announces to Mrs.\ Rudzińska that she cannot teach her daughter any longer.
Although both Mrs.\ Rudzińska and her daughter are aware of the teacher's incompetence, they do not do anything about it, knowing that without the meager income from these classes, the teacher would end up with no means of livelihood.
They offer to do anything to help the poor widow find another way to earn money.
Aleksander, Maria's cousin, having witnessed the situation, mentions that the magazine that Mr.\ Rudziński works for is currently looking for an illustrator, and suggests that Marta could apply for that position.
He goes to visit Mr.\ Rudziński at the editor's office.
Marta leaves without accepting payment for the month of French classes, saying that she has not taught little Jadwiga anything.
She feels ashamed and feels it immoral to take money for useless work.

While waiting for any response from Mr.\ Rudziński, Marta visits the office of Mrs.\ Żmińska, the tutoring broker who helped her come in contact with Mrs.\ Rudzińska.
Hoping to find another opportunity to make a living teaching French, Marta offers to teach absolute beginners, which she reckons would be a better fit for her skillset.
The broker meets her with formal coldness, deeply disappointed by Marta's resignation from the first teaching position.
By showing her incompetence, she explains, she had damaged the reputation of her brokering office.
With regard to beginner clasess, she says, the supply is much higher than the demand, and therefore she cannot promise to find any matching students.
Marta realizes that she is just one of dozens of unqualified young women who visit the office every day, each with her own set of troubles, and that there is no way the broker could possibly help them all.
She abandons all hope for a tutoring career.

Mrs.\ Rudzińska contacts Marta and notifies her that the magazine where her husband works is indeed looking for a new employee with good drawing skills.
She hands Marta a drawing and a box of drawing utensils, asking her to copy a drawing.
Whether or not she would be hired would depend entirely on her performance.
Marta does her best and brings the copied drawing back to Maria, but it immediately turns out that Marta's drawing skills and knowledge of art are not sufficient to secure an illustrator's post.
She gets feedback from the magazine telling her that she most certainly is endowed with an artistic talent, but has received insufficient education and practice.

Mrs.\ Rudzińska suggests that Marta could try and find employment as a salesperson at her friend's luxury textile shop, saying that the only qualifications necessary for the job are the knowledge of textiles and the ability to measure and cut cloth.
The owner of the shop, in spite of their old friendship, dismisses her, explaining that the job, although superficially simple, requires many skills that women do not possess, most notably the ability to keep immaculate order in the shop and in all calculations, good manners, and superb communication skills.
Unable to help Marta get a job in sales, Maria bids farewell to the poor widow, shoving an envelope filled with money into her hand, and leaves hurriedly in a carriage.
At first, Marta considers going to Mrs.\ Rudzińska's place and returning the envelope, but thinking of her malnourished child, she accepts the donation.

The following day, she walks to the shop where she used to buy her dresses during better times, asking the shop owner to hire her as a seamstress.
However, it turns out that even for that job she does not possess the necessary qualifications, not having any experience in the tailoring, and having never used a sewing machine.
Following a suggestion of an old acquaintance, Marta takes on the post of a seamstress at the sewing parlor of Mrs.\ Szwejc, a dishonest entrepreneuse who never bothered investing in sewing machines and just kept exploiting unqualified seamstresses for measly wages.


