%%% Local Variables:
%%% TeX-master: "main"
%%% TeX-command-default "LaTeX"
%%% End:

\chapter{Comparison of \textit{Marta} in Polish and Esperanto}

This chapter makes an attempt to describe the contents of the Polish original of the novel \textit{Marta} and compare it with Ludwik Zamenhof's Esperanto translation.
In the process, it should be possible to identify parts of the translation where equivalence could not be achieved and try to describe the way Zamenhof dealt with these issues in the process of his translation work.

\section{Analysis of the introduction}

The novel \textit{Marta} begins with an introductory monologue, describing the importance of love in the life of a woman.
% The introduction precedes the life story of the protagonist, Marta Świcka.
The introduction is written from the point of view of a third-person omniscient narrator.
It is not clear whether the narrator can be identified with the author of the novel, however, the opinions presented in the introduction can be interpreted as belonging to the author.
In this section, the narrator takes issue with the novel \textit{Albina} by Mr. Jan Zachariasiewicz (1825--1906).
Zachariasiewicz is attributed with stating that the reason women suffer, either on the moral or the physical level, is because they lack true love for a man, and that for a woman, the act of marriage is always motivated by pure calculation.
The narrator considers this statement to be unjust and argues that the whole existence of a woman is built around the concept of love:
Starting from early childhood, young girls yearn to grow up and have the honor to meet their destined life partner.
In some cases, their wishes are granted, and they end up marrying a man they love at the church and living happily ever after.
In many other cases, something along the way goes wrong and the woman has to live a sinful life of suffering and hunger.

In the source text, the introduction is written in a lofty, poetic style with archaic wording.
Zamenhof managed to translate the introduction with a high degree of equivalence, and to some extent made the translation easier to understand than the original.

% Syntactic parallelism

Życie kobiety to wiecznie gorejący płomień miłości --- powiadają \textbf{jedni}.
Życie kobiety to zaparcie się --- twierdzą \textbf{inni}.
Życie kobiety to macierzyństwo --- wołają \textbf{tamci}.
Życie kobiety to igraszka --- żartują \textbf{inni jeszcze}.

La vivo de virino estas eterne brulanta flamo de amo, diras \textbf{unuj}. La vivo de virino estas sinoferado, certigas \textbf{aliaj}. La vivo de virino estas patrineco, krias \textbf{parto da homoj}. La vivo de virino estas amuziĝado, ŝercas \textbf{aliaj}.

\section{Analysis of chapter one}

The first chapter of the novel describes the beginning of a new phase in the life of the protagonist, Marta Świcka, following the death of her husband.
Mrs. Świcka, having lost her only source of income, is compelled to move out of a lush, comfortable apartment in Graniczna street, in what is nowadays the strict center of Warsaw, and take care of her four-year-old daughter.
This section of the novel presents the circumstances in which the protagonist leaves the apartment, surrounded not by friends or family, but by porters in dusty boots, moving expensive furniture out of the building, and taking with them the last remnants of a life the family could no longer afford.

An important detail of the translation apparent from this section of the translation is that the name of the street Graniczna, (as well as all other street names throughout the Esperanto edition), is left in the original Polish spelling, meaningless to a reader not versed in any Slavic language. % TODO
Apart from that, throughout this section the translator managed to preserve both the precise meaning of most of the words and their literary value, rendering the translated text easy to read and understand.

--- Mamo! --- szepnęła dziewczynka --- patrz! biurko ojca!

--- Panjo! --- mallaŭte diris la knabineto, --- la skribtablo de la patro!

Znać było, że przedmioty te, z którymi \textbf{rozstawała się widocznie}, posiadały dla niej nie tylko materialną cenę;

Oni povis rimarki, ke tiuj objektoj, kiuj \textbf{videble estis forprenataj de ŝi}, havis por ŝi valoron ne sole materialan;

In the whole section, the Polish verb \textit{szeptać} (`to whisper') is always translated into Esperanto as \textit{mallaŭte paroli} (`to speak quietly').
Both expressions convey similar meaning and though not fully equivalent, this change is unlikely to result in mistranslations in the Chinese text.

Po biurku ukazał się na dziedzińcu ładny kralowski fortepian, ale kobieta w żałobie obojętniej już za nim wzrokiem powiodła.

Post la skribtablo aperis sur la korto bela fortepiano, sed la virino en funebra vesto akompanis ĝin jam per rigardo pli indiferenta.

The expression \textit{kralowski fortepian} (`a grand piano made by the company Krall i Seider') has been translated as \textit{bela fortepiano} (`a beautiful grand piano'), omitting the indication of origin altogether.
The said company was a popular maker of pianofortes in Poland during the life and times of Orzeszkowa, but the company name was unlikely to be recognized abroad.
Therefore, the omission seems reasonable and was not likely to result in mistranslations.

--- Łóżeczko moje, mamo! --- zawołała dziewczynka --- ludzie ci i łóżeczko moje zabierają, i tę kołderkę, którąś mi sama zrobiła! \textbf{Ja nie chcę, aby oni to zabierali!} Odbierz, mamo, od nich łóżeczko moje i kołderkę.

--- Mia liteto, panjo! --- ekkriis la knabineto --- tiuj homoj forprenas ankaŭ mian liteton, kaj ankaû tiun kovrileton, kiun vi mem al mi faris! Reprenu, panjo, de ili mian liteton kaj la kovrileton.

In this fragment, the sentence \textit{Ja nie chcę, aby oni to zabierali!} (I do not want them to take it away) is omitted altogether.

--- Dziękuję ci, \textbf{Zosiu} --- rzekła cicho --- byłaś dla mnie bardzo dobrą.

--- Mi dankas vin, \textbf{Sonjo}, ŝi diris mallaŭte, --- vi estis por mi tre bona.

In the source text, the protagonist refers to her former servant using the name ``Zosia,'' which is a diminutive form of the name Zofia.
Although there is no clear rule regarding diminutive names in Esperanto, Zamenhof nativized the name as ``Sonjo.''
This may be the Esperanto variant of the name ``Sonja,'' which is the most common diminutive form of the analogous name ``Sofia'' in the Russian language.
