\section{Summary of the plot of chapter two}

Then follows a brief biography of the protagonist preceding the beginning of the first chapter.
Born to a family of gentry and married to a low-ranking government official by the name Jan Świcki, she lost first her mother, then her father, and finally her husband.
During her husband's disease, Marta had to spend most of their money and sell off her jewellery to pay for medical costs, which turned out to be useless.

On the next day of the removal, Marta goes to the office of a tutoring broker in an attempt to apply for a job.
At the office, she encounters an English woman and a French woman.
Both are offered very well-paid jobs as governesses staying at the residences of wealthy counts, with very good working conditions.
The French woman is even allowed to live with her little niece.
The broker admits that the French woman does not possess any particular teaching qualifications, but has the benefit of being a foreigner.
After a brief interview with the broker, a middle-aged lady, she finds out that she has little hope of finding a teaching job and staying together with her daughter; she could, however, get a decently paid job if she agreed to leave the child to stay with someone else.
Not willing to part with her daughter, Marta offers to provide entry-level classes in such subjects as geography, history, Polish literature, and drawing.
She finds out that she cannot teach these subjects, as they are taught almost exclusively by men.
The broker admonishes Marta that in their society, the only way a woman could earn a decent living with her work is by attaining mastery in a highly demanded skill, such as a foreign language or music; any elementary skills are not enough to secure a carefree existence.
