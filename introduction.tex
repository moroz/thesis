\spacing{1.32}
\chapter{Introduction}

The purpose of this thesis is to discuss the 1873 novel \textit{Marta}, written by the Polish writer Eliza Orzeszkowa (1841-1910), as well as its translations into Esperanto and Traditional Chinese.
Eliza Orzeszkowa 

The Chinese version was translated from the Esperanto translation in May 1928 by Mr. Zhong Xianmin (錘憲民 \nazwisko{Zhōng Xiànmín}) and first published in 1930 by the Shanghai Beixin Shuju publishing house (上海北新書局 \nazwisko{Shànghǎi Běixīn Shūjú}).
(Orzeszko 1948).

All Chinese terms mentioned in this thesis are listed in Traditional Chinese characters, together with the \textit{Hanyu Pinyin} romanization (漢語拼音 \pinyin{Hànyǔ Pīnyīn}) with tone marks. Of those, the romanizations of common words is set in italic type, while proper names are set in regular font.

\section{The life and times of Eliza Orzeszkowa}

In order to fully understand the works of Eliza Orzeszkowa, it is important to put her writings in their historical context.
Poland did not exist as a sovereign state after 1795, when the remaining territory of the Polish-Lithuanian Commonwealth was annexed by the Russian Empire, the  Kingdom of Prussia, and the Austro-Hungarian Empire.
After the partitions, the ruling forces exercised policies aiming to uproot all expressions of Polish patriotism and nationalism. These policies were particularly strict in the Russian and Prussian partitions.
For 123 years between 1795 and 1918, the Polish people's struggle for independence constituted an important topic in Polish-language literature.
It was particularly important in the works of the Positivist writers, 

Eliza Orzeszkowa was born Eliza Pawłowska on June 6th, 1841, in the village of Milkowszczyzna in present-day Belarus, to a family of gentry. Her father, Benedykt Pawłowski, died when she was three years old. 
At the age of ten, she moved to Warsaw, the present-day capital of Poland, where for the following five years she attended a boarding school run by the nuns of the Order of the Holy Sacrament.
In 1858, at the age of 17, she married Piotr Orzeszko, a landowner. In many Slavic languages, including Polish, family names take a different form when referring to a man, his daughter, and his wife, the form ``Orzeszkowa'' meaning `Mrs. Orzeszko'.
At the ti
The marriage of Mr. and Mrs. Orzeszko proved unsuccesful, mainly due to the Eliza's passionate interest for the Polish 
(Britannica, Bachórz).

\section{Names of Eliza Orzeszkowa}
% 孀婦 shuang1fu4
In Chinese-language literature, Eliza Orzeszkowa is referred to by many slightly different names.
In 
(known in Chinese literature under either the transcribed name 艾麗查・奧熱什科娃 \nazwisko{Àilìchá Àorèshíkēwā}, also written 艾麗查·奧若什科娃 \nazwisko{Àilìchá Àoruòshíkēwā} or the abbreviated variant 奥西斯歌 \nazwisko{Àoxīsīgē}) was 