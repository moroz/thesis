\chapter{Introduction}
\defspacing

The purpose of this thesis is to discuss the 1873 novel \textit{Marta}, written by the Polish writer Eliza Orzeszkowa (1841-1910), as well as its translations into Esperanto and the indirect translation from Esperanto into Traditional Chinese.
The novel was first published in Polish in 1873, in Esperanto in 1810, and in Traditional Chinese in 1948.
The Chinese version was translated in May 1928 by Mr. Zhong Xianmin (錘憲民 \nazwisko{Zhōng Xiànmín}) from the Esperanto edition.
His translation was first published in 1930 by the Shanghai Beixin Shuju publishing house (上海北新書局 \nazwisko{Shànghǎi Běixīn Shūjú})
(Kökény and Bleier 1933: 612; Orzeszko 1948).

\section{A note on transcriptions and transliterations}

All Chinese terms mentioned in this thesis are listed in Traditional Chinese characters, together with the \textit{Hanyu Pinyin} romanization (漢語拼音 \pinyin{Hànyǔ Pīnyīn}) with tone marks. Of those, the romanizations of common words are set in italic type, while proper names are set in regular font. Russian terms are rendered in original Cyryllic script together with their transliteration according to the BGN/PCGN romanization system for Russian.

\section{Historical background of the novel}

In order to fully understand the works of Eliza Orzeszkowa, it is important to put her writings in their historical context.
Poland did not exist as a sovereign state after 1795, when the remaining territory of the Polish-Lithuanian Commonwealth was annexed by the Russian Empire, the  Kingdom of Prussia, and the Austro-Hungarian Empire.
After the partitions, the ruling forces exercised policies aiming to uproot any expression of Polish patriotism and nationalism. These policies were particularly strict in the Russian and Prussian partitions.
For 123 years, between 1795 and 1918, the Polish people's struggle for independence constituted an important topic in Polish-language literature.
It was particularly important in the works of Positivist writers, who believed that the key to independence was education and hard work, rather than uprisings and revolutions. (TODO: source?)

Gloger (2007) states that in the case of Poland, Positivism played a similar role to that of Enlightenment in Western Europe, that is to say, it paved ground for the development of modernism and modernity, popularizing rationalism and the scientific approach to reality. The cultural impact of Enlightenment in Poland was rather limited due to the overall civilizational lag and unfavorable historical circumstances.

\section{The life and times of Eliza Orzeszkowa}

Eliza Orzeszkowa was born Eliza Pawłowska on June 6th, 1841, in the village of Milkowszczyzna in present-day Belarus, to a family of gentry. Her father, Benedykt Pawłowski, died when she was three years old. 
At the age of ten, she moved to Warsaw, the present-day capital of Poland, where for the following five years she attended a boarding school run by the nuns of the Order of the Holy Sacrament.
In 1858, at the age of 17, she married Piotr Orzeszko, a landowner.
The marriage of Mr. and Mrs. Orzeszko proved unsuccessful, mainly due to Eliza's political interests---Eliza was highly pro-independence and sought the independence of serfs (Britannica, Bachórz).

\section{Names of Eliza Orzeszkowa}
% 孀婦 shuang1fu4
In many Slavic languages, including Polish, many family names have traditionally taken a different form when referring to a man, his unmarried daughter, and his wife.
For instance, the wife of the well-known Polish poet of the Romanticist period, Adam Mickiewicz, was referred to as Celina Mickiewiczowa, and his eldest daughter would be referred to as Maria Mickiewiczówna, up until her marriage.
Analogously, in case of the name ``Eliza Orzeszkowa,'' Orzeszkowa means `the wife of Mr. Orzeszko,' which roughly corresponds to the English form ``Mrs. Piotr Orzeszko.''
In modern-day Polish, however, this convention is virtually obsolete.
All family members use the same form of the name, and the wife of Mr. Orzeszko can be called Mrs. Orzeszko.
A notable exception to this rule are the surnames ending in \textit{-ski} or similar suffixes, such as Kowalski, Górecki, Grodzki.
These surnames have evolved from adjectives and still follow all the grammar rules pertaining to adjectives.
Thus, the wife and daughter of the Polish counterpart of John Smith, Jan Kowalski, would still use the form \textit{Kowalska}. 

In Zamenhof's translation of \textit{Marta}, the author's name is listed as Eliza Orzeszko.
Zhong Xianmin's translation follows this convention, rendering the name into Chinese with slightly distorted pronunciation as 愛麗莎・奧西斯哥 (\nazwisko{Àilìshā Àoxīsīgē}).
Other Chinese-language writings and websites use other variations, based on the traditional form ``Eliza Orzeszkowa.''
These names include 艾麗查・奧熱什科娃 (\nazwisko{Àilìchá Àorèshíkēwā}) and 艾麗查・奧若什科娃 (\nazwisko{Àilìchá Àoruòshíkēwā}). In the remaining part of this thesis, the name ``Eliza Orzeszkowa'' shall be used, being the most prevalent form in Polish-language writings.

\section{The Esperanto movement}

Esperanto (Chinese: 世界語 \pinyin{Shìjièyǔ}, lit. `world language') is a constructed language created by Ludwik Lejzer Zamenhof (Esperanto: Ludoviko Lazaro Zamenhof, 1859-1917), a Jewish ophthalmologist.
Zamenhof was born in the city of Białystok, Congress Poland, Russian Empire (present-day Białystok, Poland), an ethnically and linguistically diverse city inhabited by Poles, Russians, Germans, and Jews.
% Each and all of these nations spoke their own language and despised the others for spea
Having been raised as an idealist, in the belief that all men were brothers, he had noticed the discrepancy between his ideals and the discriminative, exclusive attitudes of the peoples inhabiting his home town.
Ever since his early childhood, he had a vision of a single language that could unite all nations (Zamenhof 1904).

The language he authored was first described in a book published in the Russian language in 1887 in Warsaw. The title of the English edition was \textit{Dr. Esperanto's International Language, Introduction and Complete Grammar} (Pre-reform Russian: Между\-на\-род\-ный Языкъ. Предисловие и полный учебникъ. \textit{Mezhdunarodnyy Yazyk. Predisloviye i polnyy uchebnik.} `The International Language. Introduction and Complete Textbook').
The book contained a basic course of Zamenhof's constructed language together with a dictionary and is commonly referred to as \textit{The First Book} (Esperanto: \textit{Unua Libro}).
In the book, Zamenhof postulated the need for an international auxilliary language that could connect people from different cultural and language backgrounds.
He argued that if the humanity had to only learn two languages, their own native language and the proposed bridge language, they would be able to communicate with each other with more ease, enriching all languages and allowing for a better command of one's own native language (Zamenhof 2006).

Similar endeavors had been undertaken before, with the most prominent example being Volapük, designed by a Catholic priest from Germany by the name Johann Martin Schleyer, however none of them had gained any significant international attention, nor had succeeded in becoming a generally accepted \textit{lingua franca}.
Their failure, Zamenhof argued, had been due to their failure to meet three crucial conditions:
% For a constructed language to become successful, Zamenhof suggested three key conditions that ought to be met. These conditions were:

\begin{enumerate}
  \item that the language be easy to learn, so as to make its acquisition a ``mere play to the learner,''
  \item designating the language as a means of international communication rather than a ``universal'' language, and enabling the learner to make direct use of his knowledge of the language with persons of any nationality,
  \item convincing indifferent people around the world to learn the proposed international language (Zamenhof 2006).
\end{enumerate}

At the time, the language had no name of its own other than the titular ``international language'' (Esperanto: \textit{lingvo internacia}).
The speakers of the new language soon decided to baptize the language with a part of Zamenhof's pseudonym, \textit{Doktoro Esperanto}.
In the constructed language, \textit{Esperanto} signified `the one who hopes.' The suffix \textit{-anto} indicates the gerund form, analogous to such Latin words as \textit{memorandum} or the ``verb + \textit{-ing}'' form in English.

Nowadays, Esperanto is often thought of merely as a Quixotic experiment.
It has never quite succeeded in becoming a commonly accepted lingua franca, nor could it possibly put an end to all wars---the turbulent history of the 20th century proves otherwise.
On the other hand, it is by far the most successful constructed language in history, with the estimated number of speakers ranging from 100,000 to 2 million, depending on the source.
Esperanto is actively spoken not only by L2 speakers, but quite often also by their bilingual or multilingual offspring.
In Esperanto, native speakers of Esperanto are referred to as \textit{denaskuloj} (from \textit{denask-} `from birth' and \textit{ulo} `person, individual') or \textit{denaskaj Esperantistoj} (`Esperantist from birth')
(Britannica 2019a).

Arika Okrent, the author of a book on the topic of constructed languages, is generally critical of most of these projects.
Nevertheless, in an interview with Jason Zesky (2009), she acknowledged the ease of communication that the speakers of Esperanto managed to attain:

\longquote{At the Esperanto conference I attended, more than I expected. I thought it would be a lot of play acting, like [in a singsong voice], “Hello. How are you? I am fine.” But they were speaking fluently, and with a little bit of study I could understand what was going on. There was also a lot of Victorian rigmarole because Esperanto was a turn of the [20th] century phenomenon they have retained all these old rituals. They have a flag passing ceremony and the reading of greetings from various Esperanto clubs—even a show and piano recitals.}

\section{The content of the novel}
\textit{Marta} tells the story of Marta {nazwisko}.
In the beginning of the narrative, Marta is presented as a relatively affluent lady whose husband had just died, leaving her with a little daughter. % has?
Due to these unfavorable circumstances, she is compelled to move out of the lavish apartment in Warsaw that she and her family had inhabited.

The novel \textit{Marta} is divided into nine unnumbered chapters and an introduction. 