\chapter{Introduction}
\defspacing

\section{Research topic and objective}
The purpose of this thesis is to discuss the 1873 novel \textit{Marta}, written by the Polish writer Eliza Orzeszkowa (1841-1910), as well as its translations into Esperanto and the indirect translation from Esperanto into Traditional Chinese.
The goal of the research is to find the similarities and differences between all three versions, analyze the translation techniques used by the two translators.
Through this process, it should be possible to determine whether any content has been distorted in the process of indirect translation from Esperanto into Chinese, and conversely, if Esperanto can reliably be used as an intermediate language.

Being himself a proponent of the Esperanto movement, the author hopes to prove the usefulness of Esperanto in the field of translation, and by extension, to promote the study and use of this language.

\section{Overview of literature and references}
In the thesis, three editions of the novel \textit{Marta} are discussed, in three different languages: Polish, Esperanto, and Traditional Chinese.
The novel was first published in Polish in 1873, in Esperanto in 1910, and in Traditional Chinese in 1948.
Ludwik Zamenhof, the creator of Esperanto, % being fluent in Polish,
translated the novel from the Polish original.
The Chinese version was translated in May 1928 by Mr. Zhong Xianmin (錘憲民 \nazwisko{Zhōng Xiànmín}%
\footnote{All Chinese terms mentioned in this thesis are listed in Traditional Chinese characters, together with the \textit{Hanyu Pinyin} romanization (漢語拼音 \pinyin{Hànyǔ Pīnyīn}) with tone marks. The romanizations of common terms are set in italic type, while proper names are set in regular font.}) %
from Zamenhof's Esperanto translation.
Zhong Xianmin's translation was first published in 1930 by the Shanghai Beixin Shuju publishing house (上海北新書局 \nazwisko{Shànghǎi Běixīn Shūjú})
(Kökény and Bleier 1933: 612; Orzeszko 1948).

None of the 

\section{Research method and expected outcome}
A comparative analysis of a single text translated into both Esperanto and through Esperanto into Chinese could determine whether Esperanto can reliably be used as an intermediate language for the translation of literary texts.

The author deems it safe to assume that the translator of the Esperanto edition, Ludwik Lejzer Zamenhof, possessed all means of creating a complete and faithful translation: having grown up in an area where Polish was the language of intelligentsia, he had a good command of the Polish language and a good understanding of the Polish culture.
He also knew the author of the novel in person and had lived in the area where the plot of the novel was placed.
Being the creator of the Esperanto language, he knew the language better than anyone else at his time.
At the same time, Esperanto is a simple and highly flexible language, with grammar and logic based on European languages, including Polish and the grammatically similar Russian, making it relatively easy to translate the Polish original into Esperanto with little to no loss of meaning and high degrees of equivalence.

On the other hand, Chinese, especially in its written variant (書面語 \pinyin{shūmiànyǔ} `book language'), relies on entirely different patterns than European languages.
This has to do with the history and development of the Chinese language.
Up till the 20th century, Chinese was a diglossic language.
No legal standard of the spoken language existed and numerous spoken varieties were spoken throughout China.
Those local varieties are referred to in Chinese as \textit{fangyan} (方言 \pinyin{fāngyán} `local language'), a word commonly rendered in English as ``dialect.''
By Western linguistic standards, these varieties of the language could not be considered dialects of a single language, being to a large extent mutually unintelligible.
Mair (1991) proposed the adoption of the term ``topolect'' as a translation of \textit{fargyan}.

Despite the vast differences between topolects, the writing system was largely uniform throughout China, and has also been used in international correspondence between various countries of the Sinitic cultural circle, particularly in Japan, Korea, and Vietnam.
In Chinese, this written variety is commonly referred to as \textit{wenyanwen} (文言文 \textit{wényánwén}), and in English as ``classical Chinese'' or ``literary Chinese.''
The language had undergone very little change between Han dynasty (漢朝 \pinyin{Hàncháo}) and the 20th century, effectively preserving a substantial part of cultural references, vocabulary, and idiomatic expressions.
% A corpus of ancient text and Confucian classics written in this language has for many centuries formed the basis of a system of formal education and imperial examinations for civil servants, which had not been abolished until as late as 1905.
Many idiomatic expressions originating from classical texts, the so-called \textit{chengyu} (成語 \pinyin{chéngyǔ}), are commonly used to precisely convey emotions and moral concepts, or as set phrases in formal correspondence.
The knowledge of various \textit{chengyu} is considered to be an indicator of one's erudition, and the use of these expressions is an important differentiating factor between colloquial Chinese and its written counterpart.

Zhong Xianmin, who translated the novel \textit{Marta} did so based only on the Esperanto edition, without any knowledge of the Polish language.
Due to these specifics of written Chinese, certain differences are to be expected between the Chinese translation and the Polish original.
Should the Esperanto translation prove to be a faithful rendition of the Polish original, it follows that any substantial differences between the texts have been introduced in Zhong Xianmin's work.

\section{Historical background of the novel}

In order to fully understand the works of Eliza Orzeszkowa, it is important to put her writings in their historical context.
Poland did not exist as a sovereign state after 1795, when the remaining territory of the Polish-Lithuanian Commonwealth was annexed by the Russian Empire, the  Kingdom of Prussia, and the Austro-Hungarian Empire.
After the partitions, the ruling forces exercised policies aiming to uproot any expression of Polish patriotism and nationalism. These policies were particularly strict in the Russian and Prussian partitions.

For 123 years, between 1795 and 1918, the Polish people's struggle for independence constituted an important topic in Polish-language literature.
After the failure of the 1863 January Uprising against the Russian Empire, the Polish people were disappointed with Romanticism and slogans of armed fight for independence.
The fallen uprising was an important milestone in the development of socialism.
It was in honor of the fallen uprising that a worker gathering was held in 1864, attended by Marx himself, resulting in the founding of the International Workingmen's Association, also known as the ``First International.''

The end of the January uprising is considered to be the beginning of a literary genre known as the ``Warsaw Positivism'' (French: \textit{positivisme varsovien}).
% It was particularly important in the works of Positivist writers, who believed that the key to independence was education and hard work, rather than uprisings and revolutions.
The key topics in Polish Positivist literature included the fight for independence, the emancipation of serfs and women, promoting science, medicine and public hygiene, and the assimilation of Jews into the Polish society.
(TODO: source?)

Gloger (2007) states that in the case of Poland, Positivism played a similar role to that of Enlightenment in Western Europe, that is to say, it paved ground for the development of modernism and modernity, popularizing rationalism and the scientific approach to reality. The cultural impact of Enlightenment in Poland was rather limited due to the overall civilizational lag and unfavorable historical circumstances.

\section{The life and times of Eliza Orzeszkowa}

Eliza Orzeszkowa was born Eliza Pawłowska on June 6th, 1841, in the village of Milkowszczyzna in present-day Belarus, to a family of gentry. Her father, Benedykt Pawłowski, died when she was three years old. 
At the age of ten, she moved to Warsaw, the present-day capital of Poland, where for the following five years she attended a boarding school run by the nuns of the Order of the Holy Sacrament, where she learned French and German, and explored Polish literature.
In 1858, at the age of 17, her parents arranged her marriage with Piotr Orzeszko, a wealthy landowner.
At the day of their wedding, the bridegroom was 35 years old.
The marriage of Mr. and Mrs. Orzeszko proved unsuccessful, mainly due to Eliza's political interests---Eliza was highly pro-independence and sought the emancipation of serfs (Britannica, Bachórz, Brykowisko 2011).

During the January Uprising, Eliza was actively working for the Polish cause, passing messages between the troops, and even helping Mr. Romuald Traugutt, the leader of the insurrection from October 1863 up to its end in August 1864, by hiding him in her house and escorting him to the border of Congress Poland.
In December 1864, Mr. Orzeszko was arrested and sent to Russia.
(Brykowisko 2011).

In 1866, Eliza settled in Grodno. In 1869, her marriage was annulled.

\section{Names of Eliza Orzeszkowa}

In many Slavic languages, including Polish, many family names have traditionally taken a different form when referring to a man, his unmarried daughter, and his wife.
For instance, the wife of the well-known Polish poet of the Romanticist period, Adam Mickiewicz, was referred to as Celina Mickiewiczowa, and his eldest daughter would be referred to as Maria Mickiewiczówna, up until her marriage.
Analogously, in case of the name ``Eliza Orzeszkowa,'' Orzeszkowa means `the wife of Mr. Orzeszko,' which roughly corresponds to the English form ``Mrs. Piotr Orzeszko.''
In modern-day Polish, however, this convention is virtually obsolete.
All family members use the same form of the name, and the wife of Mr. Orzeszko can be called Mrs. Orzeszko.
A notable exception to this rule are the surnames ending in \textit{-ski} or similar suffixes, such as Kowalski, Górecki, Grodzki.
These surnames have evolved from adjectives and still follow all the grammar rules pertaining to adjectives.
Thus, the wife and daughter of the Polish counterpart of John Smith, Jan Kowalski, would still use the form \textit{Kowalska}. 

In Zamenhof's translation of \textit{Marta}, the author's name is listed as Eliza Orzeszko.
Zhong Xianmin's translation follows this convention, rendering the name into Chinese with slightly distorted pronunciation as 愛麗莎・奧西斯哥 (\nazwisko{Àilìshā Àoxīsīgē}).
Other Chinese-language writings and websites use other variations, based on the traditional form ``Eliza Orzeszkowa.''
These names include 艾麗查・奧熱什科娃 (\nazwisko{Àilìchá Àorèshíkēwā}) and 艾麗查・奧若什科娃 (\nazwisko{Àilìchá Àoruòshíkēwā}). In the remaining part of this thesis, the name ``Eliza Orzeszkowa'' shall be used, being the most prevalent form in Polish-language writings.

\section{The origins of Esperanto}

Esperanto (Chinese: 世界語 \pinyin{Shìjièyǔ}, lit. `world language') is a constructed language created by Ludwik Lejzer Zamenhof (Esperanto: Ludoviko Lazaro Zamenhof, 1859-1917), a Jewish ophthalmologist.
Zamenhof was born in the city of Białystok, Congress Poland, Russian Empire (present-day Białystok, Poland), an ethnically and linguistically diverse city inhabited by Poles, Russians, Germans, and Jews.
% Each and all of these nations spoke their own language and despised the others for spea
Having been raised as an idealist, in the belief that all men were brothers, he had noticed the discrepancy between his ideals and the discriminative, exclusive attitudes of the peoples inhabiting his home town.
Ever since his early childhood, he had a vision of a single language that could unite all nations (Zamenhof 1904).

The father and grandfather of Ludwik Zamenhof had been language teachers, and Ludwik himself had a deep interest for languages since early childhood, dreaming of becoming a great Russian poet.
At the age of ten, he wrote a five-act tragedy.
Between 1870 he enrolled at a gymnasium in Białystok, which he attended for nine years.
In 1873, he moved with his parents to Warsaw, where he studied Latin and Ancient Greek, and then enrolled at the Philological Gymnasium (a high school with a particular emphasis on the study of languages).
He graduated in 1879 and moved to Moscow to attend the faculty of medicine.
In 1881, due to the poor financial situation of his family, he was compelled to move back to Warsaw, where he obtained his medical degree in 1885.
After a period of medical practice he realized that he was too sympathetic for his patients, that their suffering and death affected him too much.
This was the reason he became a specialist in ophthalmology
(Kökény and Bleier 1933: 1048).

The language he authored was first described in a book published in the Russian language in 1887 in Warsaw. The title of the English edition was \textit{Dr. Esperanto's International Language, Introduction and Complete Grammar} (Pre-reform Russian: Между\-на\-род\-ный Языкъ. Предисловие и полный учебникъ. \textit{Mezhdunarodnyy Yazyk. Predisloviye i polnyy uchebnik.}%
\footnote{All Russian terms in the thesis are rendered in original Cyryllic script together with their transliteration according to the BGN/PCGN romanization system for Russian.}
`The International Language. Introduction and Complete Textbook').
The book contained a basic course of Zamenhof's constructed language together with a brief dictionary and is commonly referred to as \textit{The First Book} (Esperanto: \textit{Unua Libro}).
In the book, Zamenhof postulated the need for an international auxilliary language that could connect people from different cultural and language backgrounds.
He argued that if the humanity had to only learn two languages, their own native language and the proposed bridge language, they would be able to communicate with each other with more ease, enriching all languages and allowing for a better command of one's own native language. The book was subsequently published in Polish, German, and French editions (Zamenhof 2006).
% TODO: Find a source for Schleyer's inability to converse in Volapük

\section{Esperanto and Volapük}
Zamenhof was by no means the first man to construct an auxilliary language with the hope of unifying the human race.
Many similar endeavors had been undertaken before, with the most prominent example being Volapük, designed by a Catholic priest from Germany by the name Johann Martin Schleyer.
None of those constructed languages had gained any significant international attention or had succeeded in becoming a generally accepted \textit{lingua franca}.
Their failure, Zamenhof argued, had been due to their failure to meet three crucial conditions:

\begin{enumerate}
  \item that the language be easy to learn, so as to make its acquisition a ``mere play to the learner,''
  \item designating the language as a means of international communication rather than a ``universal'' language, and enabling the learner to make direct use of his knowledge of the language with persons of any nationality,
  \item convincing indifferent people around the world to learn the proposed international language (Zamenhof 2006).
\end{enumerate}

Schleyer made all efforts to make his language as comprehensive and precise as possible.
He considered Volapük to be his language and property, which to some extent may have hindered the development of the language.
The vocabulary of Volapük was based mostly on English, with some influences of German and French.
Most of the loanwords deviated so much from their respective source languages that they were beyond easy recognition by speakers of these languages.
The language was difficult to understand for anyone without prior training, while the distortions obfuscated the European origin of the language and made it language equally easy---or equally hard---to non-Europeans as to Europeans.
It is said that even Schleyer himself could not express his thoughts clearly in Volapük
% A common point in the criticisms of Schleyer's creation was the use of umlauts, derived from Schleyer's native German language.
(Kökény and Bleier 1933: 1012).

By contrast, Zamenhof's approach to language design was brilliantly simple: the first book included only 16 grammar rules, which have been left intact ever since, and a vocabulary of just 917 stems.
His vocabulary of Esperanto was mostly based on Latin and Western European languages, mainly French and German.
Zamenhof avoided adding too detailed explanations in the belief that the language would eventually develop on its own
(Kökény and Bleier 1933: 1053-1054).

\section{Early development of Esperanto}

In the early stages of the development of Esperanto, the language had no name of its own other than the titular ``international language'' (Esperanto: \textit{lingvo internacia}).
The speakers of the new language soon decided to baptize the language with a part of Zamenhof's pseudonym, \textit{Doktoro Esperanto}.
In the constructed language, \textit{Esperanto} signified `the one who hopes.' The suffix \textit{-anto} indicates the gerund form, analogous to such Latin words as \textit{memorandum} or the ``verb + \textit{-ing}'' form in English.
A person who learns and speaks Esperanto is referred to as an \textit{Esperantist} (Esperanto: \textit{Esperantisto}, from \textit{Esperanto} + \textit{-isto}, suffix indicating a person who does something over a longer period of time, analogous to \textit{-ist} in European languages).

After a period of natural development of the language, in 1905, Zamenhof compiled another work called \textit{Fundamento de Esperanto} (English: \textit{Foundation of Esperanto}), in which he included a more detailed grammar, exercises, and an extended set of vocabulary, in five national languages: French, English, German, Russian, and Polish.
This book was designated as the only obligatory authority over the language
(Kökény and Bleier 1933: 1053-1054).

\section{Esperanto today}
Nowadays, Esperanto is often thought of merely as a Quixotic experiment.
It has never quite succeeded in becoming a commonly accepted lingua franca, nor could it possibly put an end to all wars---the turbulent history of the 20th century proves otherwise.
On the other hand, it is by far the most successful constructed language in history, with the estimated number of speakers ranging from 100,000 to 2 million, depending on the source.
Esperanto is actively spoken not only by L2 speakers, but quite often also by their bilingual or multilingual offspring.
In Esperanto, native speakers of Esperanto are referred to as \textit{denaskuloj} (from \textit{denask-} `from birth' and \textit{ulo} `person, individual') or \textit{denaskaj Esperantistoj} (`Esperantist from birth')
(Britannica 2019a).

There are numerous websites and organizations that are, to this day, actively promoting Esperanto and providing Esperanto learning resources free of charge.
Notable examples of such websites include Lernu.net\footnote{\url{https://lernu.net/en}, retrieved March 23th, 2020.} and Duolingo\footnote{\url{https://www.duolingo.com/course/eo/en/Learn-Esperanto}, retrieved March 23th, 2020.}.
As of September 12th, 2019, the Esperanto edition of Wikipedia, the free online encyclopedia, included 276,488 articles\footnote{\url{https://eo.m.wikipedia.org/wiki/Vikipedio_en_Esperanto}, retrieved March 23th, 2020.}.
The 105th World Esperanto Congress (Esperanto: \textit{105-a Universala Kongreso de Esperanto}) is scheduled to take place from 1st to the 8th of August, 2020, in Montreal, Quebec, Canada\footnote{\url{https://esperanto2020.ca/en/world-esperanto-congress/}, retrieved March 23th, 2020.}.
The 104th World Esperanto Congress took place in Lahti, Finland, and attracted 917 participants from 57 countries (Universala Esperanto-Asocio 2019).

Arika Okrent, the author of a book on the topic of constructed languages, is generally critical of most of these projects.
Nevertheless, in an interview with Jason Zesky (2009), she acknowledged the ease of communication that the speakers of Esperanto managed to attain:

\longquote{%
  {\bfseries Jason Zesky: I understand you've been to several invented language conferences. What do people do at these conferences?}

  Arika Okrent: At the Esperanto conference I attended, more than I expected. I thought it would be a lot of play acting, like [in a singsong voice], ``Hello. How are you? I am fine.'' But they were speaking fluently, and with a little bit of study I could understand what was going on. % There was also a lot of Victorian rigmarole because Esperanto was a turn of the [20th] century phenomenon they have retained all these old rituals. They have a flag passing ceremony and the reading of greetings from various Esperanto clubs—even a show and piano recitals.%
}

\section{The content of the novel \textit{Marta}}
\textit{Marta} tells the story of Marta Swicka.
In the beginning of the narrative, Marta is presented as a relatively affluent, twenty-odd-year-old lady whose husband had just died, leaving her with a little daughter, no living family members, and no means of livelihood.
Due to these unfavorable circumstances, she is compelled to move out of a lavish apartment in Warsaw to a plain, dilapidated .

The novel \textit{Marta} is divided into nine unnumbered chapters and an introduction. 