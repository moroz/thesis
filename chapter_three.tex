\chapter{Presentation of the novel \textit{Marta}}

\section{Historical background of the novel}

In order to fully understand the works of Eliza Orzeszkowa, it is important to put her writings in their historical context.
Poland did not exist as a sovereign state after 1795, when the remaining territory of the Polish-Lithuanian Commonwealth was annexed by the Russian Empire, the  Kingdom of Prussia, and the Austro-Hungarian Empire.
After the partitions, the ruling forces exercised policies aiming to uproot any expression of Polish patriotism and nationalism. These policies were particularly strict in the Russian and Prussian partitions.

For 123 years, between 1795 and 1918, the Polish people's struggle for independence constituted an important topic in Polish-language literature.
After the failure of the 1863 January Uprising against the Russian Empire, the Polish people were disappointed with Romanticism and slogans of armed fight for independence.
The fallen uprising and the Polish question had also been an important topic for European socialists
(Stekloff 1928).

The end of the January uprising is considered to be the beginning of a literary genre known as the ``Warsaw Positivism'' (French: \textit{positivisme varsovien}).
This philosophy emphasized the importance of ``organic work'' (Polish: \textit{praca organiczna}), that is, active development of education and economy.
% It was particularly important in the works of Positivist writers, who believed that the key to independence was education and hard work, rather than uprisings and revolutions.
The key topics in Polish Positivist literature included the fight for independence, the emancipation of serfs and women, promoting science, medicine and public hygiene, and the assimilation of Jews into the Polish society.
Besides Eliza Oreszkowa, important representants of Warsaw Positivism included Bolesław Prus (1847-1912), Maria Konopnicka (1842-1910), and Henryk Sienkiewicz (1846-1916), the laureate of the Nobel Prize in Literature 1905
(eSzkola.pl).

Gloger (2007) states that in the case of Poland, Positivism played a similar role to that of Enlightenment in Western Europe, that is to say, it paved ground for the development of modernism and modernity, popularizing rationalism and the scientific approach to reality. The cultural impact of Enlightenment in Poland was rather limited due to the overall civilizational lag and unfavorable historical circumstances.

\section{The life and times of Eliza Orzeszkowa}

Eliza Orzeszkowa was born Eliza Pawłowska on June 6th, 1841, in the village of Milkowszczyzna in present-day Belarus, to a family of gentry. Her father, Benedykt Pawłowski, died when she was three years old. 
At the age of ten, she moved to Warsaw, the present-day capital of Poland, where for the following five years she attended a boarding school run by the nuns of the Order of the Holy Sacrament, where she learned French and German, and explored Polish literature.
In 1858, at the age of 17, her parents arranged her marriage with Piotr Orzeszko, a wealthy landowner.
At the day of their wedding, the bridegroom was 35 years old.
The marriage of Mr. and Mrs. Orzeszko proved unsuccessful, mainly due to Eliza's political interests---Eliza was highly pro-independence and sought the emancipation of serfs (Britannica, Bachórz, Brykowisko 2011).

During the January Uprising, Eliza was actively working for the Polish cause, passing messages between the troops, and even helping Mr. Romuald Traugutt, the leader of the insurrection from October 1863 up to its end in August 1864, by hiding him in her house and escorting him to the border of Congress Poland.
In December 1864, Mr. Orzeszko was arrested and sent to Russia.
(Brykowisko 2011).

In 1866, Eliza settled in Grodno. In 1869, her marriage was annulled.

\section{Names of Eliza Orzeszkowa}

In many Slavic languages, including Polish, many family names have traditionally taken a different form when referring to a man, his unmarried daughter, and his wife.
For instance, the wife of the well-known Polish poet of the Romanticist period, Adam Mickiewicz, was referred to as Celina Mickiewiczowa, and his eldest daughter would be referred to as Maria Mickiewiczówna, up until her marriage.
Analogously, in case of the name ``Eliza Orzeszkowa,'' Orzeszkowa means `the wife of Mr. Orzeszko,' which roughly corresponds to the English form ``Mrs. Piotr Orzeszko.''
In modern-day Polish, however, this convention is virtually obsolete.
All family members use the same form of the name, and the wife of Mr. Orzeszko can be called Mrs. Orzeszko.
A notable exception to this rule are the surnames ending in \textit{-ski} or similar suffixes, such as Kowalski, Górecki, Grodzki.
These surnames have evolved from adjectives and still follow all the grammar rules pertaining to adjectives.
Thus, the wife and daughter of the Polish counterpart of John Smith, Jan Kowalski, would still use the form \textit{Kowalska}. 

In Zamenhof's translation of \textit{Marta}, the author's name is listed as Eliza Orzeszko.
Zhong Xianmin's translation follows this convention, rendering the name into Chinese with slightly distorted pronunciation as 愛麗莎・奧西斯哥 (\nazwisko{Àilìshā Àoxīsīgē}).
Other Chinese-language writings and websites use other variations, based on the traditional form ``Eliza Orzeszkowa.''
These names include 艾麗查・奧熱什科娃 (\nazwisko{Àilìchá Àorèshíkēwā}) and 艾麗查・奧若什科娃 (\nazwisko{Àilìchá Àoruòshíkēwā}). In the remaining part of this thesis, the name ``Eliza Orzeszkowa'' shall be used, being the most prevalent form in Polish-language writings.
\section{The content of the novel \textit{Marta}}
\textit{Marta} tells the story of Marta Swicka.
In the beginning of the narrative, Marta is presented as a relatively affluent, twenty-odd-year-old lady whose husband had just died, leaving her with a little daughter, no living family members, and no means of livelihood.
Due to these unfavorable circumstances, she is compelled to move out of a lavish apartment in Warsaw to a plain, dilapidated .

The novel \textit{Marta} is divided into nine unnumbered chapters and an introduction. 