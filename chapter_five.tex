\chapter{Comparison of \textit{Marta} in Chinese and Esperanto}

This chapter aims to analyze the indirect translation of the novel \textit{Marta} into Chinese based on the Esperanto translation by Zamenhof.
Through a careful analysis of the differences, the Author hopes to identify mistranslations, fragments of the text translated with little equivalence, as well as other discrepancies in meaning from the Polish original.
Based on those differences, it should be possible to determine the reason for said differences, be it a result of earlier mistranslations, linguistic choices made by Zamenhof, or shortcomings of the international language itself.

\section{Comparison of the first chapter}

\subsection{Alterations in sentence structure}

Starting from the first chapter, there are several places where the sentence structure has been obviously altered to better fit the grammar and usage of the Chinese language.

許多穿著不雅觀的衣服和骯髒的靴子的人,從一座通二層的梯上走下來。

\begin{displayquote}
De la pura, larĝa ŝtuparo, kiu kondukis al la pli alta etaĝo de la konstruaĵo, konstante deiradas homoj en malelegantaj vestoj kaj malelegantaj polvokovritaj botoj.
\end{displayquote}
