\usepackage[no-math]{fontspec}
\usepackage[usenames,dvipsnames,svgnames,table]{xcolor}
\usepackage[compact]{titlesec}
\usepackage[all]{nowidow}
\usepackage{setspace,url,indentfirst,hyphenat,xeCJK,makeidx,sectsty,fancyhdr,csquotes,emptypage}
\usepackage[top=2.5cm,left=3.5cm,right=2.5cm,bottom=2.5cm]{geometry}
\setmainfont[Mapping=times]{Adobe Garamond Pro}
\setsansfont[Scale=0.88]{Myriad Pro}
\setCJKmainfont[Scale=0.89]{Source Han Serif TC}
\setCJKsansfont{Source Han Sans TW}
\setCJKmonofont{Source Han Sans TW}
\newfontfamily\pyfont{Adobe Text Pro}
\newcommand{\toponim}[1]{\nohyphens{\pyfont #1}}
\newcommand{\nazwisko}[1]{\nohyphens{\pyfont #1}}
\newcommand{\pinyin}[1]{\nohyphens{\pyfont\itshape #1}}
\newcommand{\defspacing}{\spacing{1.32}}
\newcommand{\quotespacing}{\spacing{1.1}}
\let\fnm=\footnotemark
\pagestyle{fancy}
\fancyhf{}
\fancyhead[LE]{\nouppercase\leftmark}
\fancyhead[RO]{\nouppercase\rightmark}
\fancyfoot[CE,CO]{\thepage}
\renewcommand{\headrulewidth}{0pt}
\newcommand{\ibid}{ (\textit{Ibidem}).}
\newcommand{\longquote}[1]{\quotespacing\blockquote{\itshape #1}\defspacing}
\urlstyle{sf}

\author{Karol Moroz 莫若思}
\title{A Comparative Analysis of the Esperanto and Chinese Translations of \textit{Marta} by Eliza Orzeszkowa}
\newcommand{\titlezh}{奥西斯歌《孤雁淚》的波蘭文原文版及其世界語、中文譯本的比較}