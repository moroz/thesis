\newcommand{\toponim}[1]{\nohyphens{\pyfont #1}}
\newcommand{\nazwisko}[1]{\nohyphens{\pyfont #1}}
\newcommand{\pinyin}[1]{\nohyphens{\pyfont\itshape #1}}
\newcommand{\ipa}[1]{{\ipafont #1}}
\newcommand{\defspacing}{\spacing{1.5}}
\let\fnm=\footnotemark
\newcommand{\timestamp}{Compiled {\ddmmyyyydate\today} at \currenttime}
\newcommand{\ibid}{ (\textit{Ibidem}).}
\newcommand{\longquote}[1]{\quotespacing\blockquote{\itshape #1}\defspacing}

\newcommand{\noteonrussian}{\footnote{All Russian terms in the thesis are rendered in original Cyryllic script and transliterated according to the BGN/PCGN romanization system for Russian. Unless noted otherwise, all terms are listed in their pre-reform spelling, the standard form before the October Revolution of 1917.}}

\makeatletter
%Take the original environment definition and change the leftmargin to 1cm
\renewenvironment*{displayquote}
  {\begingroup\setlength{\leftmargini}{1cm}\csq@getcargs{\csq@bdquote{}{}}}
  {\csq@edquote\endgroup}
\makeatother
\renewcommand{\mkbegdispquote}{\onehalfspacing\itshape}
