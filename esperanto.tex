\chapter{The Esperanto movement}

Esperanto (Chinese: 世界語 \pinyin{Shìjièyǔ}, lit. `world language') is a constructed language created by Ludwik Lejzer Zamenhof (Esperanto: Ludoviko Lazaro Zamenhof, 1859-1917), a Jewish ophthalmologist from Białystok.
This chapter deals with the constructed language Esperanto.
It briefly describes the life story of Ludwik Zamenhof, the creator of Esperanto, the development of his language and the Esperanto movement, and the state of the Esperanto movement to date.

\section{The origins of Esperanto}
Ludwik Lejzer Zamenhof was born on December 15th, 1859, in the city of Białystok, Congress Poland, Russian Empire (present-day Białystok, Poland), an ethnically and linguistically diverse city inhabited by Poles, Russians, Germans, and Jews.
% Each and all of these nations spoke their own language and despised the others for spea
Having been raised as an idealist, in the belief that all men were brothers, he had noticed the discrepancy between his ideals and the discriminative, exclusive attitudes of the peoples inhabiting his home town.
Ever since his early childhood, he had a vision of a single language that could unite all nations (Zamenhof 1904).

The father and grandfather of Ludwik Zamenhof had been language teachers, and Ludwik himself had a deep interest for languages since early childhood, dreaming of becoming a great Russian poet.
At the age of ten, he wrote a five-act tragedy.
Between 1870 he enrolled at a gymnasium in Białystok, which he attended for nine years.
In 1873, he moved with his parents to Warsaw, where he studied Latin and Ancient Greek, and then enrolled at the Philological Gymnasium (a high school with a particular emphasis on the study of languages).
He graduated in 1879 and moved to Moscow to attend the faculty of medicine.
In 1881, due to the poor financial situation of his family, he was compelled to move back to Warsaw, where he obtained his medical degree in 1885.
After a period of medical practice he realized that he was too sympathetic for his patients, that their suffering and death affected him too much.
This was the reason he became a specialist in ophthalmology, a relatively peaceful branch of medicine.
This was also the time when he adopted his pseudonym, ``Doktoro Esperanto.''
In the constructed language, \textit{Esperanto} signified `the one who hopes,'
the suffix \textit{-anto} indicating the gerund form, analogous to such Latin words as \textit{memorandum} or the ``verb + \textit{-ing}'' form in English
(Kökény and Bleier 1933: 1048).

\subsection{First appearance of Esperanto}
The language authored by Zamenhof was first described in a book published in the Russian language in 1887 in Warsaw.
The first book in its Russian edition was titled \textit{Mezhdunarodnyy Yazyk. Predisloviye i polnyy uchebnik.}
(Pre-reform Russian: Между\-на\-род\-ный Языкъ. Предисловие и полный учебникъ%
\footnote{All Russian terms in the thesis are rendered in original Cyryllic script and transliterated according to the BGN/PCGN romanization system for Russian.} `The International Language. Introduction and Complete Textbook').
This book has subsequently been translated into English as \textit{Dr. Esperanto's International Language, Introduction and Complete Grammar}.
The book contained a basic course of Zamenhof's constructed language together with a brief dictionary and is commonly referred to as \textit{The First Book} (Esperanto: \textit{Unua Libro}).
In the book, Zamenhof postulated the need for an international auxilliary language that could connect people from different cultural and language backgrounds.
He argued that if the humanity had to only learn two languages, their own native language and the proposed bridge language, they would be able to communicate with each other with more ease, enriching all languages and allowing for a better command of one's own native language. The book was subsequently published in Polish, German, and French editions (Zamenhof 2006).
% TODO: Find a source for Schleyer's inability to converse in Volapük

\subsection{Esperanto and Volapük}
Zamenhof was by no means the first man to construct an auxilliary language with the hope of unifying the human race.
Many similar endeavors had been undertaken before, with the most prominent example being Volapük, designed by a Catholic priest from Germany by the name Johann Martin Schleyer.
None of those constructed languages had gained any significant international attention or had succeeded in becoming a generally accepted \textit{lingua franca}.
Their failure, Zamenhof argued, had been due to their failure to meet three crucial conditions:

\begin{enumerate}
  \item that the language be easy to learn, so as to make its acquisition a ``mere play to the learner,''
  \item designating the language as a means of international communication rather than a ``universal'' language, and enabling the learner to make direct use of his knowledge of the language with persons of any nationality,
  \item convincing indifferent people around the world to learn the proposed international language (Zamenhof 2006).
\end{enumerate}

Schleyer made all efforts to make his language as comprehensive and precise as possible.
He considered Volapük to be his language and property, which to some extent may have hindered the development of the language.
The vocabulary of Volapük was based mostly on English, with some influences of German and French.
Most of the loanwords deviated so much from their respective source languages that they were beyond easy recognition by speakers of these languages.
The language was difficult to understand for anyone without prior training, while the distortions obfuscated the European origin of the language and made it language equally easy---or equally hard---to non-Europeans as to Europeans.
It is said that even Schleyer himself could not express his thoughts clearly in Volapük
% A common point in the criticisms of Schleyer's creation was the use of umlauts, derived from Schleyer's native German language.
(Kökény and Bleier 1933: 1012).

By contrast, Zamenhof's approach to language design was brilliantly simple: the first book included only 16 grammar rules, which have been left intact ever since, and a vocabulary of just 917 stems.
His vocabulary of Esperanto was mostly based on Latin and Western European languages, mainly French and German.
Zamenhof avoided adding too detailed explanations in the belief that the language would eventually develop on its own
(Kökény and Bleier 1933: 1053-1054).

\subsection{Early development of Esperanto}

In the early stages of the development of Esperanto, the language had no name of its own other than the titular ``international language'' (Esperanto: \textit{lingvo internacia}).
The speakers of the new language soon decided to baptize the language with a part of Zamenhof's pseudonym, \textit{Doktoro Esperanto}.
A person who learns and speaks Esperanto is referred to as an \textit{Esperantist} (Esperanto: \textit{Esperantisto}, from \textit{Esperanto} + \textit{-isto}, suffix indicating a person who does something over a longer period of time, analogous to \textit{-ist} in European languages).

After a period of natural development of the language, in 1905, Zamenhof compiled another work called \textit{Fundamento de Esperanto} (English: \textit{Foundation of Esperanto}), in which he included a more detailed grammar, exercises, and an extended set of vocabulary, in five national languages: French, English, German, Russian, and Polish.
This book was designated as the only obligatory authority over the language
(Kökény and Bleier 1933: 1053-1054).

\subsection{Esperanto in China}
According to Boltinsky (2016), the most famous proponent of Esperanto in China had been the writer and poet Lu Xun (魯迅 \nazwisko{Lǔ xùn}, 1881-1936), an important figure in the May Fourth Movement (五四運動 \pinyin{Wǔ sì yùndòng}) that advocated the transition from literary to vernacular Chinese.

Liu (2016) names two periods in Chinese history during which Esperanto had enjoyed relative popularity.
The first period had been between 1912-1936, that is, roughly between the establishment of the Republic of China and the escalation of the Second Sino-Japanese War (抗日戰爭 \pinyin{kàngrì zhànzhēng}) to the whole of China.
During that period, Esperanto enjoyed support of many Chinese intellectuals, including the president of Peking University (北京大學 \pinyin{Běijīng Dàxué}), Cai Yuanpei (蔡元培 \nazwisko{Cài Yuánpéi}, 1868-1940).
Another period of relative popularity followed the ``reform and opening up'' (改革開放 \pinyin{gǎigé kāifàng}) program of Deng Xiaoping (鄧小平 \nazwisko{Dèng Xiǎopíng}, 1904-1997), and occurred roughly between 1981-2005.
Liu admits that although Esperantists in China constitute an important part of the global Esperanto movement, the number of Esperanto speakers in this country is vastly exaggerated.
At present, there is no nationwide organization for Esperantists in China, only a handful of local Esperantist clubs.
Liu estimates the number of proficient speakers in the whole country to be around 1000.

Nowadays, there are three news outlets in the People's Republic of China that publish in Esperanto, namely the website \textit{El Popola Ĉinio}\footnote{\url{http://www.espero.com.cn/}, retrieved March 24th, 2020.} (`From People's China'), China.org.cn\footnote{\url{http://esperanto.china.org.cn/}, retrieved March 24th, 2020.},
and China Radio International\footnote{\url{http://esperanto.cri.cn/}, retrieved March 24th, 2020.}
(Boltinsky 2016).

\subsection{Esperanto today}
Nowadays, Esperanto is often thought of merely as a Quixotic experiment.
It has never quite succeeded in becoming a commonly accepted lingua franca, nor could it possibly put an end to all wars---the turbulent history of the 20th century proves otherwise.
On the other hand, it is by far the most successful constructed language in history, with the estimated number of speakers ranging from 100,000 to 2 million, depending on the source.
Esperanto is actively spoken not only by L2 speakers, but quite often also by their bilingual or multilingual offspring.
In Esperanto, native speakers of Esperanto are referred to as \textit{denaskuloj} (from \textit{denask-} `from birth' and \textit{ulo} `person, individual') or \textit{denaskaj Esperantistoj} (`Esperantist from birth')
(Britannica 2019a).

There are numerous websites and organizations that are actively promoting Esperanto and providing Esperanto learning resources free of charge.
Notable examples of such websites include Lernu.net\footnote{\url{https://lernu.net/en}, retrieved March 23th, 2020.}, offering online courses for self-study, e-books, and an electronic dictionary, dedicated exclusively to Esperanto; and Duolingo\footnote{\url{https://www.duolingo.com/course/eo/en/Learn-Esperanto}, retrieved March 23th, 2020.}, offering interactive courses of several languages, including Esperanto.

As of September 12th, 2019, the Esperanto edition of Wikipedia, the free online encyclopedia, included 276,488 articles\footnote{\url{https://eo.m.wikipedia.org/wiki/Vikipedio_en_Esperanto}, retrieved March 23th, 2020.}.
The 105th World Esperanto Congress (Esperanto: \textit{105-a Universala Kongreso de Esperanto}) is scheduled to take place from 1st to the 8th of August, 2020, in Montreal, Quebec, Canada\footnote{\url{https://esperanto2020.ca/en/world-esperanto-congress/}, retrieved March 23th, 2020.}.
The 104th World Esperanto Congress took place in Lahti, Finland, and attracted 917 participants from 57 countries (Universala Esperanto-Asocio 2019).

Arika Okrent, the author of a book on the topic of constructed languages, is generally critical of most of these projects.
Nevertheless, in an interview with Jason Zesky (2009), she acknowledged the ease of communication that the speakers of Esperanto managed to attain:

\longquote{%
  {\bfseries Jason Zesky: I understand you've been to several invented language conferences. What do people do at these conferences?}

  Arika Okrent: At the Esperanto conference I attended, more than I expected. I thought it would be a lot of play acting, like [in a singsong voice], ``Hello. How are you? I am fine.'' But they were speaking fluently, and with a little bit of study I could understand what was going on. % There was also a lot of Victorian rigmarole because Esperanto was a turn of the [20th] century phenomenon they have retained all these old rituals. They have a flag passing ceremony and the reading of greetings from various Esperanto clubs—even a show and piano recitals.%
}

